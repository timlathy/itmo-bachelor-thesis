\documentclass[a4paper,14pt]{extarticle}
\usepackage{fontspec, unicode-math}
\usepackage[english, russian]{babel}
\setmainfont{Times New Roman}
\setmonofont{CMU Typewriter Text}

% === Formatting ===
% Based on https://github.com/3ap/ifmo-vkr-preamble

\usepackage[top=20mm, bottom=20mm, left=25mm, right=10mm]{geometry}

\usepackage[nodisplayskipstretch]{setspace}
\onehalfspacing

\addto\captionsrussian{
  \renewcommand{\contentsname}{Оглавление}
}

\usepackage{indentfirst}
\setlength{\parindent}{1.25cm}

\usepackage{titlesec}
\titleformat{\section}[block]{\centering\bfseries\large}{\arabic{section}}{1ex}{\MakeUppercase}
\titleformat{\subsection}[block]{\hspace{\parindent}\bfseries\normalsize}{\arabic{section}.\arabic{subsection}}{1ex}{}
\titleformat{\subsubsection}[block]{\hspace{\parindent}\bfseries\normalsize}{\arabic{section}.\arabic{subsection}.\arabic{subsubsection}}{1ex}{}
\titlespacing*{\section}{0pt}{42pt}{42pt}

% === Commands ===

% \newcommand...

% === Body ===

\begin{document}

\tableofcontents
\newpage

\section*{ВВЕДЕНИЕ}
\addcontentsline{toc}{section}{ВВЕДЕНИЕ}

\textbf{Актуальность темы исследования.} Графические процессоры широко используются для
математических расчетов, в частности, в машинном обучении. Для достижения оптимальной
утилизации аппаратных ресурсов алгоритмы реализуются при помощи ассемблера. Отсутствие
средств верификации ассемблерных программ для ГП затрудняет разработку и
может привести к некорректным результатам.

В работе рассматриваются графические процессоры AMD, которые характеризуются несовместимостью
архитектуры команд между семействами ГП, что повышает необходимость в статической
верификации кода.\newline

\textbf{Степень теоретической разработанности темы.} 

\textbf{Цель работы:} (...)

\textbf{Задачи:}
\begin{itemize}
\item ?
\end{itemize}

\textbf{Практическая значимость:} (...)

\section{Обзор предметной области}

\subsection{Средства анализа программ}

Статический и динамический анализ. Динамический анализ затрудняется тем, что некорректное
поведение может вызвать аппаратную ошибку, восстановление которой потребует перезагрузки
системы и приведет к потере данных для анализа.  % citation needed

Как правило, инструменты статического анализа работают с исходным кодом программ,
поскольку в них сохраняются высокоуровневые конструкции (переменные, циклы, функции).

В ассемблерных программах подобные конструкции реализуются при помощи макросов,
которые не несут в себе полезной для анализа информации (области видимости и т.д.).
Помимо этого, существует несколько трансляторов ассемблерного кода — GAS, CRLX, проприетарные — которые
различаются набором макросов и синтаксисом инструкций.

В связи с этим в качестве исходных данных в работе рассматриваются исполняемые файлы,
формат которых строго определен.

\subsection{Программы для графических процессоров}

SIMT, workitems, wavefronts, workgroups.

Существующие средства статического анализа для шейдеров направлены на обнаружение
состояний гонки (\textit{data races}), при которых несколько потоков записывают
и читают одни и те же данные в памяти без синхронизации.
% sci-hub.do/10.1007/978-3-319-08867-9_15

Аппаратный набор инструкций графических процессоров NVIDIA недоступен для разработчиков.
Вместо этого предоставляется виртуальный набор инструкций PTX, который остается
обратно совместимым между семействами ГП и транслируется драйвером перед выполнением.
% https://docs.nvidia.com/pdf/ptx_isa_7.0.pdf (goals of ptx)

Компания AMD предоставляет доступ к набору инструкций каждого семейства ГП. Таким образом,
программист получает полный контроль над исполнением программ, в том числе становится
ответственным за соблюдение аппаратных ограничений.

В связи с этим, при рассмотрении статическиго анализа исполняемых программ, имеет
смысл сосредоточиться на ГП AMD.

\subsection{Исполнение программ на графических процессорах AMD}

При написании ассемблерных программ для графических процессоров AMD программисту
необходимо вручную разрешать зависимости по данным и управлению. Задача усложняется
% https://developer.amd.com/wp-content/resources/Vega_Shader_ISA_28July2017.pdf
тем, что ситуации, которые необходимо обрабатывать вручную, меняются от семейства к семейству ГП
и могут быть незадокументированными в официальных документах.
% https://gitlab.freedesktop.org/mesa/mesa/blob/a0c003007515e31c63e18f4a3c8abc86814143bf/src/amd/compiler/README

\end{document}
